\documentclass[]{elsarticle} %review=doublespace preprint=single 5p=2 column
%%% Begin My package additions %%%%%%%%%%%%%%%%%%%
\usepackage[hyphens]{url}

  \journal{Advances in Investment Analysis and Portfolio Management} % Sets Journal name


\usepackage{lineno} % add
\providecommand{\tightlist}{%
  \setlength{\itemsep}{0pt}\setlength{\parskip}{0pt}}

\usepackage{graphicx}
\usepackage{booktabs} % book-quality tables
%%%%%%%%%%%%%%%% end my additions to header

\usepackage[T1]{fontenc}
\usepackage{lmodern}
\usepackage{amssymb,amsmath}
\usepackage{ifxetex,ifluatex}
\usepackage{fixltx2e} % provides \textsubscript
% use upquote if available, for straight quotes in verbatim environments
\IfFileExists{upquote.sty}{\usepackage{upquote}}{}
\ifnum 0\ifxetex 1\fi\ifluatex 1\fi=0 % if pdftex
  \usepackage[utf8]{inputenc}
\else % if luatex or xelatex
  \usepackage{fontspec}
  \ifxetex
    \usepackage{xltxtra,xunicode}
  \fi
  \defaultfontfeatures{Mapping=tex-text,Scale=MatchLowercase}
  \newcommand{\euro}{€}
\fi
% use microtype if available
\IfFileExists{microtype.sty}{\usepackage{microtype}}{}
\bibliographystyle{elsarticle-harv}
\ifxetex
  \usepackage[setpagesize=false, % page size defined by xetex
              unicode=false, % unicode breaks when used with xetex
              xetex]{hyperref}
\else
  \usepackage[unicode=true]{hyperref}
\fi
\hypersetup{breaklinks=true,
            bookmarks=true,
            pdfauthor={},
            pdftitle={Financialization and Commodity Price Volatility: The Case of Grains},
            colorlinks=false,
            urlcolor=blue,
            linkcolor=magenta,
            pdfborder={0 0 0}}
\urlstyle{same}  % don't use monospace font for urls

\setcounter{secnumdepth}{0}
% Pandoc toggle for numbering sections (defaults to be off)
\setcounter{secnumdepth}{0}
% Pandoc header



\begin{document}
\begin{frontmatter}

  \title{Financialization and Commodity Price Volatility: The Case of Grains}
    \author[strathclyde]{Devraj Basu}
   \ead{devraj.basu@strath.ac.uk} 
  
    \author[strathclyde]{Olivier Bauthéac}
   \ead{olivier.bautheac@strath.ac.uk} 
  
    \author[thomas]{Ameeta Jaiswal-Dale}
   \ead{a9jaiswal@stthomas.edu} 
  
      \address[strathclyde]{University of Strathclyde, Glasgow, UK}
    \address[thomas]{University of St.~Thomas, Minneapolis, US}
  
  \begin{abstract}
  The early 2000s have witnessed a dramatic increase in long-only
  institutional investments in commodity markets. This surge and its
  accompanying effects are commonly referred to as the financialization of
  commodity markets. This paper studies the impact of this phenomenon on
  price volatility in the grain markets where we focus on CBOT corn, wheat
  \& soybeans and KCBOT wheat. Our results suggest that increases in
  trading volume and open interest, a consequence of financialization,
  appear to have changed the nature of grains volatility and seem
  consistent with the model of Stein (1987) and Goldstein and Yang (2015)
  where the entry of new traders could lower the information content of
  price for existing traders. Our findings further suggest that the
  increase in market depth has a generally destabilizing effect on grains
  volatility which provides some support for the concerns of regulators.
  However this destabilizing effect does not seem to be driven by the
  action of speculators. Our analysis is thus overall more supportive of
  Singleton (2013) and Stein (1987) in that disagreement and difference of
  opinion are more likely to have caused changes in the nature of grains
  volatility than excess speculation.
  \end{abstract}
  
 \end{frontmatter}

\newpage

\hypertarget{introduction}{%
\section{Introduction}\label{introduction}}

In the early 2000s, against a backdrop of a low yield environment and
poor stock market performance, the investment industry developed
financial products designed for providing individuals and institutions
with buy-side exposure to commodities through over-the-counter (OTC)
swaps, exchange-traded funds (ETFs), and exchange-traded notes (ETNs),
all of which are linked to popular commodity indexes such as the Goldman
Sachs Commodity Index (S\&P-GSCI). As these products grew in popularity
investment in long-only commodity index funds rapidly soared\footnote{Assets
  allocated to commodity index replication strategies grew from \$13bn
  in 2003 to over \$300bn in 2008 (Masters and White, 2011).}. Some
refer to this large inflow of mostly institutional capital and its
impact as the financialization of commodity markets (Domanski and Heath,
2007). The issue has had a wide impact on areas ranging from
financial\footnote{For example Tang and Xiong (2012), Singleton (2013),
  Basak and Pavlova (2016), Henderson et al. (2015).} and
agricultural\footnote{Irwin and D. R. Sanders (2012a) study the impact
  on agricultural markets, Irwin and Sanders (2011) and Hamilton and Wu
  (2015) on commodity markets in general, Büyükşahin and Robe (2014)
  study the oil market while Korniotis and others (2009) considers the
  metals market.} economics to public policy\footnote{The first
  responses to the 2007/2008 crisis of escalating food and energy prices
  took the form of policy reports, many of which reasoned that the
  growth of commodity index funds came along with an influx of largely
  speculative capital that was responsible for driving commodity prices
  beyond their historic highs (De Schutter, 2010; Herman et al., 2011;
  Schumann, 2011; Senate, 2009; UNCTAD and Cooperation, 2009).} and
revived the long-standing ``adequacy of speculation'' debate.
Agricultural commodities have been the forefront of the controversy over
whether ``excess speculation'', allegedly brought about by
financialization, has contributed to price spikes in commodity
markets\footnote{Irwin and Sanders (2011); Rouwenhorst and Tang (2012);
  Cheng and Xiong (2014a) and Bos and Molen (2012) survey the literature
  and summarize the policy and academic debates.}. On one side,
championed by hedge fund manager Michael Masters, are those who argue
that index investor driven buying pressure created a massive bubble in
commodity prices\footnote{This contention is commonly referred to as the
  Masters' Hypothesis.}, particularly in the grain and energy markets
(Caballero et al., 2008; Du and Zhao, 2017; Hamilton, 2009; Masters,
2008; Masters and White, 2011; Petzel, 2009). Others, advocated by the
academic duo formed by Dwight Sanders and Scott Irwin, are dismissive of
this contention and point out inconsistencies as well as contradictory
facts in the bubble arguments (Harris and Buyuksahin, 2009; Irwin et
al., 2009; Korniotis and others, 2009; Krugman, 2008; Pirrong, 2010,
2008; Sanders and Irwin, 2008; Stoll and Whaley, 2010; Till, 2009). A
number of academic studies attempted to sort out which side of the
debate is correct using a variety of economic tools\footnote{While
  Gilbert (2010a), Gilbert (2010b), Phillips et al. (2011), Phillips and
  Yu (2011), Tang and Xiong (2012) are supportive of the bubble
  argument, Harris and Buyuksahin (2009), Brunetti and Buyuksahin
  (2009), Sanders et al. (2010), Stoll and Whaley (2010), Sanders and
  Irwin (2010), Sanders et al. (2010), Sanders and Irwin (2011), Irwin
  and D. R. Sanders (2012b), Büyükşahin and Robe (2014), Kilian and
  Murphy (2014) dismiss it.}. The majority of these studies does not
support, and some of them even refute, the bubble hypothesis suggesting
that there is no direct link between commodity institutional investments
and commodity prices. Nonetheless, the impact of financialization on
commodity price volatility, in the grain markets in particular, is still
a source of concern both from an academic as well as a regulatory
perspective\footnote{Concerns over the consequences of financialization
  were behind Rule 76 FR 4752 issued by the U.S. Commodity Futures
  Trading Commission (CFTC) on January 26, 2011. This provision emanates
  from the Dodd-Frank Wall Street and Consumer Protection Act of 2010
  (Title VII, Section 737) that mandates the CFTC to use position limits
  to restrict the flow of speculative capital into a number of commodity
  markets. The Rule was approved in a close 3-2 vote and the ensuing
  rule-making process was extremely contentious with several
  commissioners expressing reservations about the lack of supporting
  evidence and the Rule also triggering thousands of comment letters as
  well as a lawsuit against the CFTC. See remarks of ex CFTC Chairman
  Gary Gensler before the International Monetary Fund Conference
  (\href{http://www.cftc.gov/PressRoom/SpeechesTestimony/opagensler-137}{www.cftc.gov})
  as well as remarks of Commissioner Bart Chilton
  (\href{http://www.cftc.gov/PressRoom/SpeechesTestimony/chiltonstatement022412}{www.cftc.gov}).}
and has not been thoroughly investigated yet\footnote{Bohl and Stephan
  (2013) is one of the few studies that investigate this issue.}.

In this study we examine the nature of this impact\footnote{Our paper
  fits into the broad area of modeling changes in volatility and
  volatility transmission. See for example: Gannon (2010), Jiang et al.
  (2017), Li (2016), Fung et al. (2003), Smith and Bracker (2003),
  Nishina et al. (2012).}. We focus on Chicago Board of Trade (CBOT)
corn, soybeans and soft red winter wheat (SRW) as well as Kansas City
Board of Trade (KCBOT) hard red winter wheat (HRW), four major global
commodities of which the U.S. are a major producer. As financialization
is largely a U.S. based phenomenon it seems appropriate to study its
effect on them. Besides, these markets are regarded as classic hedging
markets where speculation tends to follow hedging volume (Sanders and
Baker, 2012; Working, 1962, 1960, 1954, 1953) and are thus good
candidates for assessing the impact of financialization, where
speculation plays a central role. Also, the prices of these commodities
tend to be driven by similar fundamentals as they can be substitutes
and/or complements on the production and use sides. This makes it easier
to isolate the effect of financialization by examining the differences
in volatility patterns before and during financialization. Finally,
corn, soybeans, HRW and SRW wheat are constituents of major commercial
commodity indexes and are thus particularly suitable in this context.

We study the volatility of the futures front month returns as well as
that of the basis for each individual commodity considered over the
1992-2007 period using a set of volatility estimators that includes the
classic ``close-to-close'' as well as a number of range-based estimators
that account for intra-day price action. We define the 1992-2003 period
as the pre-financialization phase and the 2003-2007 as the
financialization phase with the 2003 cut-off based on earlier
studies\footnote{Most earlier studies locate the onset of
  financialization around the 2003-2004 period (Basak and Pavlova, 2016;
  Cheng and Xiong, 2014a; Hamilton and Wu, 2015; Irwin and D. R.
  Sanders, 2012a, 2012b; Irwin and Sanders, 2011; Tang and Xiong, 2012).}.
We find a moderate increase in futures average volatility (from 10\% to
25\% depending on the estimator) and a much larger increase in basis
average volatility (from 30\% to over 100\% depending on the estimator).
Although uniform, the increase in average futures volatility is perhaps
not as high as proponents of the Masters' Hypothesis might have believed
while that for basis volatility suggests potentially stronger effects
due to financialization.

The relationship between volatility, volume and open interest has been
investigated quite extensively in the finance literature with some
studies suggesting that increases trading volume and open interest, via
the entry of new traders, could have a stabilizing effect on volatility
(Bessembinder and Seguin, 1993; Kyle, 1985; Stoll and Whaley, 1987)
while others, particularly in some recent studies, argue it could have a
destabilizing effect (Goldstein and Yang, 2015; Singleton, 2013; Sockin
and Xiong, 2015). We investigate the issue by running regressions of
volatility on contemporaneous and lagged volume and open interest.

For futures volatility we observe an increase in \(R^{2}\) over the
financialization period in most cases, sometimes dramatically\footnote{The
  increase is largest for corn, followed by soybeans and HRW wheat.
  Results are mixed for SRW wheat.} as well as an overall increase in
slope coefficient\footnote{For SRW and HRW wheat both open interest and
  volume switch from negative (as predicted by Kyle (1985), Stoll and
  Whaley (1987)) to positive (as predicted by most informational models;
  Karpoff (1987)) during financialization. For corn, open interest
  follows the same pattern while volume shows a positive relationship
  other the first period which gets stronger during financialization.
  Soybeans show the opposite pattern with a negative correlation for
  both open interest and volume which becomes more negative during
  financialization.} for both open interest and volume, contemporaneous
and lagged. The pattern of results is similar for basis volatility.
These results suggest that the relationship between volatility and both
volume and open interest has changed and taken together with the large
increase in level for both variables, suggest that financialization has
affected grains volatility. Open interest and volume seem to have
switched from a stabilizing to a destabilizing role during
financialization for corn and both HRW and SRW wheat while for soybeans,
the opposite seems to have happened. Overall our results suggest that
increases in trading volume and open interest, a consequence of
financialization, appear to have had a destabilizing influence on grain
volatility and seem consistent with the models of Stein (1987)\footnote{He
  argues that entry of new traders could lower the information content
  of price for existing traders through noise in their signals.}
Goldstein and Yang (2015)\footnote{They argue that the negative
  information content effect is caused by the behaviors of those traders
  who are informed of the same information but respond to this
  information in opposite directions. Their model suggests that
  commodity financialization could make futures prices less informative
  (they refer to price informativeness as the amount of residual
  uncertainty uninformed traders face after conditioning on prices).},
Singleton (2013)\footnote{He notes that learning about economic
  fundamentals with heterogeneous information may induce excessive price
  volatility, drift in commodity prices, and a tendency towards booms
  and busts. He argues that under these conditions the flow of financial
  index investments into commodity markets may harm price discovery and
  social welfare.} and Sockin and Xiong (2015)\footnote{They develop a
  model to analyze information aggregation in commodity markets. Their
  analysis highlights important feedback effects of informational noise
  originating from supply shocks and futures market trading on commodity
  demand and spot prices.}.

We try to assess the impact of speculators by running similar
regressions with open interest of hedgers and speculators separately as
allowed by the U.S. Commodity Futures Trading Commission (CFTC)'s
Commitment of Traders (COT) reports data\footnote{A similar sort of
  analysis using the CFTC's classification was carried out in Wang
  (2003). See the CFTC's explanatory notes for details on the traders
  classification in the COT legacy format:
  \href{http://www.cftc.gov/MarketReports/CommitmentsofTraders/ExplanatoryNotes/index.htm}{www.cftc.gov}}.
The results for speculators suggest a move towards a greater
destabilizing effect for HRW wheat, corn and SRW wheat with no clear
results for soybeans. The pattern is very similar for hedgers open
interest which raises questions about the classification\footnote{The
  CFTC has now refined its classification and publishes a
  ``disaggregated'' COT report with data going back to June 2006. In
  this report, the commercial category is further subdivided into
  processors/merchants and swap dealers while the non-commercial
  category is split into money managers and other reportables (not
  captured in the other groups). See the CFTC's explanatory notes for
  details on the traders classification in the COT disaggregated format:
  \href{http://www.cftc.gov/MarketReports/CommitmentsofTraders/DisaggregatedExplanatoryNotes/index.htm}{www.cftc.gov}}
and perhaps the nature of hedgers' activities\footnote{See Cheng and
  Xiong (2014b).} and also suggests that these may have been influenced
by financialization. The pattern of results for basis volatility is very
similar.

We investigate the speculation issue further by running similar
regressions with the Working's T index, a measure of excess
speculation\footnote{This measure originates in Working (1960) and is
  widely used in the agricultural economics literature to assess the
  impact of excess speculation.}. The results show no clear pattern
across commodities or periods with both positive and negative slope
coefficients in the second period. There is thus no clear indication
that the change in relationship between volatility and open interest has
been driven by excess speculation or an increase in speculative open
interest. We conclude with a Granger causality analysis that we
implement in turn on each pair of the above-mentioned variables. The
results are generally inconclusive with some evidence that futures
volatility and volume Granger cause each other and futures volatility
Granger causes hedger's open interest in the second period.

Overall our results point to an increase in grains volatility as well as
a clear change in the relationship between grains volatility and open
interest and volume as a result of financialization. However there is no
clear evidence that this change was driven by the actions of
speculators. Our findings provide some support for the concerns of
regulators but are overall more supportive of Goldstein and Yang (2015),
Singleton (2013) and Stein (1987) in that disagreement and difference of
opinion\footnote{This issue is also of relevance in the recent legal
  literature (Stout, 1998, p. @stout\_uncertainty\_2011) where it is
  referred to as disagreement based trading based on differing
  subjective beliefs about future prices. Stout (2011) shows how this
  may be viewed as a sort of market failure.} are more likely to have
caused changes in the nature of grains volatility than excess
speculation.

The rest of the paper is organized as follow: the data and methods are
described in the next section with the results discussed in section
\ref{results}, while section \ref{conclusions} concludes.

\newpage

\hypertarget{data-methods}{%
\section{Data \& methods}\label{data-methods}}

We study Chicago Board of Trade (CBOT) corn, soybeans and soft red
winter wheat (SRW) as well as Kansas City Board of Trade (KCBOT) hard
red winter wheat (HRW). The futures contract market quotes as well as
volume and open interest data are from Bloomberg wile spot market price
time series are from the Minneapolis Grain Exchange (MGEX).

Futures prices are observed every trading day at close while the daily
values for cash prices are constructed as the average of high and low
prices for the day due to the lack of open and close quotes in the cash
markets. We consider the front month futures contract until the first
week of the maturity month at which date the position is switched to the
next most liquid contract. For each commodity we define open interest
(volume) as the sum of all traders' positions (trading volume) for all
contracts on the term structure up to a year ahead. We define the basis
as the difference between the futures and cash (spot) price as follows:

\[B_{i, t}=F_{i, t}-S_{i, t}\] \(B_{i, t}\equiv\) basis for commodity
\textit{i}, at time \textit{t}.\\
\(F_{i, t}\equiv\) futures price for commodity \textit{i}, at time
\textit{t}.\\
\(S_{i, t}\equiv\) spot price for commodity \textit{i}, at time
\textit{t}.

We observe the 1992-2007 period and define the 1992-2003 period as the
pre-financialization phase and the 2003-2007 as the financialization
phase (with the 2003 cut-off based on earlier studies\footnote{Most
  earlier studies locate the onset of financialization around the
  2003-2004 period (Basak and Pavlova, 2016; Cheng and Xiong, 2014a;
  Hamilton and Wu, 2015; Irwin and D. R. Sanders, 2012a, 2012b; Irwin
  and Sanders, 2011; Tang and Xiong, 2012).}), and study the two periods
independently.

For futures volatility we consider a set of estimators that includes the
classic ``close-to-close'' as well as five range-based stochastic
volatility estimators: Parkinson, Garman \& Klass, Rogers \& Satchell,
Garman \& Klass-Yang \& Zhang and Yang \& Zhang. The Parkinson estimator
estimates the volatility of the underlying based on high and low prices.
The Garman \& Klass estimator assumes Brownian motion with zero drift
and no opening jumps and is 7.4 times more efficient than the
``close-to-close'' estimator. The Rogers \& Satchell estimator allows
for non-zero drift, but assumes no opening jump while the Garman \&
Klass-Yang \& Zhang estimator, a modified version of the Garman \& Klass
estimator allows for opening jumps. The Yang \& Zhang has minimum
estimation error, and is independent of drift and opening gaps. It can
be interpreted as a weighted average of the Rogers \& Satchell
estimator, the ``close-open'' volatility, and the ``open-close''
volatility. For futures the results are presented for volatility when
estimated using the classic ``close-to-close'' estimator, as it is the
most commonly used, and the Yang \& Zhang estimator as it has the
minimum estimation error, with the rest of the results available from
the authors upon request. For basis volatility, the lack of open and
close quotes restricts the set to the classic ``close-to-close'' and the
Parkinson estimators. Yet, a straightforward construction of the
Parkinson estimator is unworkable because of the few occurrences of the
zero value in the time series of the basis leading to undefined
volatility observations. As a result, we define the Parkinson estimate
of basis volatility as the difference between that of the futures and
that of the spot.

We study the relationship between volatility, volume and open interest
with a set of factor models that include combinations of the latter. We
construct four one-factor models where volatility is the response and
both contemporaneous and one-week lagged open interest and volume are in
turn the explanatory variable. We also construct four two-factor models
where volatility is the response and the explanatory variables are in
turn, contemporaneous and one-week lagged open interest, contemporaneous
and one-week lagged volume, open interest and volume as well as one-week
lagged open interest and one-week lagged volume. We conclude with a
four-factor model that includes all the above-mentioned as explanatory
variables.

The weekly CFTC COT report's format breaks down total open interest into
three categories of traders, namely hedgers (commercial), speculators
(non-commercial) and non-reportable with the latter gathering the
remaining traders who do not fit in the previous two categories. This
break down allows the construction of the Working's T index, a measure
of excess speculation, as follows:

\[
T_{i, t} =
\left\{\begin{matrix}
1+\frac{SS_{i, t}}{HS_{i, t}+HL_{i, t}} & if & HS_{i, t} \geq HL_{i, t}\\ 
1+\frac{SL_{i, t}}{HS_{i, t}+HL_{i, t}} & if & HS_{i, t} < HL_{i, t}
\end{matrix}\right.
\]

\(T_{i, t}\equiv\) Working's T index for commodity \textit{i}, at time
\textit{t}.\\
\(SS_{i, t}\equiv\) number of speculative short positions for commodity
\textit{i}, at time \textit{t}.\\
\(SL_{i, t}\equiv\) number of speculative long positions for commodity
\textit{i}, at time \textit{t}.\\
\(HS_{i, t}\equiv\) number of short hedging positions for commodity
\textit{i}, at time \textit{t}.\\
\(HL_{i, t}\equiv\) number of long hedging positions for commodity
\textit{i}, at time \textit{t}.

We refine the analysis with a set of one factor models where volatility
is the response and total open interest, open interest of hedgers, open
interest of speculators, volume and Working's T index are in turn the
explanatory variable. We conclude with two four-factor models with the
first one including the contemporaneous time series of all the above and
the second one their one-week lagged version.

We look deeper into the speculation issue with a careful Granger
causality analysis. Phillips and Loretan (1990) showed that Granger
causality testing can be unreliable in a context similar to
financialization. We hence rely on a modified version of the procedure
developed by Toda and Yamamoto (1995) that accounts for non-stationary
time series. We implement the analysis at the daily frequency with
futures volatility as measured by the Yang \& Zhang estimator, open
interest and volume as well as at the weekly frequency where we also
include open interest of hedgers and speculators separately.

Robustness considerations lead us to implement the whole analysis above
with a 1999 cut-off\footnote{Some studies date the premise of
  financialization back to the very late 1990s/early 2000s (Bohl and
  Stephan, 2013; Büyükşahin and Robe, 2014; Stoll and Whaley, 2010).}.
The pattern of results is similar although not as pronounced as with the
2003 cut-off suggesting that the effects of financialization were
strongest over the 2003-2007 period as suggested in earlier studies.

\newpage

\hypertarget{results}{%
\section{Results \& discussion}\label{results}}

Table 1 shows the mean futures front month and spot price levels, open
interest and volume for the four commodities and two periods of
interest. We observe a moderate increase of grain prices over the
financialization period with an average mean futures price increase of
11.7\%; 10.5\% for mean spot price. SRW wheat shows the strongest
increase with +17.2\% for mean futures price and +20.3\% for mean spot
price while corn shows the lowest with a 2\% and 0.8\% increase
respectively. For the basis the situation is not as clear with the wheat
complex showing a relative stability in numerical terms (+\$1.3 and
+\$0.4 respectively) while corn and soybeans show dramatic increases
(+\$6.4 or +138.5\% and +\$8.7 or +264.7\% respectively). In contrast to
prices, mean open interest and volume show a dramatic increase across
the board with average increase of 95.4\% and 63.5\% respectively
suggesting that the large increase of trading activity during
financialization did not lead to a commensurate increase in grain
prices.

Table 2 shows the average volatility levels for the four commodities and
the two periods of interest. For futures average volatility, estimators
include the classic ``close-to-close'' as well as the values for the
five range-based estimators. The pattern of results is different across
the various commodities and estimators but in all cases the average
volatility is higher over the financialization period with the average
mean increase for the set of estimators ranging from 11.3\% for SRW
wheat to 20.1\% for HRW wheat. For basis average volatility, estimators
include the classic ``close-to-close'' as well as Parkinson and show a
dramatic increase over the financialization with the average increase
for the two estimators ranging from 41.7\% for SRW wheat to 155.6\% for
corn.

Table 3 shows the results for the regressions of volatility on
contemporaneous and one-week lagged volume and open interest. For
futures volatility, the slope coefficients of open interest for corn and
both types of wheat change to positive during financialization from
negative before indicating open interest changes from having a
stabilizing effect on volatility in the first period consistent with a
classic hedging market (Working, 1962, 1960) to a destabilizing effect
in the second for these commodities. The \(R^{2}\)s vary across the two
periods with a sharp increase for corn (11\% to 44\%) and a decrease for
HRW wheat (28\% to 5\%) suggesting that while the nature of the effect
is uniform for the three commodities, the magnitude is not. For soybeans
the sign of all slope coefficients is negative in both periods and the
\(R^{2}\) is higher in the financialization period indicating a greater
stabilizing effect. The pattern and magnitude of the results is very
similar when lagged open interest is used, suggesting that it is the
expected component of open interest that is driving these results
(Bessembinder and Seguin, 1993). For volume, the slope coefficients
switch from negative to positive for SRW \& HRW wheat while, in both
periods, they are positive for corn with a substantially higher
\(R^{2}\) over the financialization period (2.8\% to 24.2\% for the Yang
\& Zhang estimator) and negative for soybeans with again a higher
\(R^{2}\) in the financialization phase. The pattern is very similar for
lagged volume indicating that the results are driven by expected volume.
The pattern of results is thus similar for both open interest and volume
and indicates that over financialization they had a destabilizing effect
on futures volatility for corn and the two types of wheat while having a
more stabilizing effect for soybeans. Besides, soybeans showed the most
pronounced increase in average futures volatility indicating that
changes in the nature of volatility and the changes in the magnitude of
volatility are not necessarily directly linked. Overall these findings
provide evidence for the hypotheses advanced in Goldstein et al. (2014),
Singleton (2013) and Stein (1987) where more trading arising from
greater dispersions of beliefs or the entry of new traders increasing
noise in signals could lead to changes in the nature of volatility and
the information content of prices. Thus greater market depth may not
always lead to more informative price signals and this may be due to the
nature of trading (see also Stout (1998) \& Stout (2011)) for a
discussion of this issue from a legal standpoint). For basis volatility
there does not appear to be a clear pattern and the results suggest that
the nature of basis volatility is different from that of futures, with
other factors perhaps driving the increases.

Combining volume and open interest produces a mixed pattern of results
as shown in table 4. For futures volatility, corn and SRW wheat show a
change in sign and increase in \(R^{2}\) (Yang \& Zhang estimator)
during financialization while for HRW wheat, there is a change in sign
for the open interest slope coefficient with the coefficient for volume
positive in both periods and a lower \(R^{2}\) for the financialization
period. These results suggest that both factors jointly had a
destabilizing influence on volatility. For soybeans, the two
coefficients are negative in both periods with a higher \(R^{2}\) for
the financialization period indicating a greater stabilizing effect. The
results for the four-factor model are shown in table 5. The clearest
pattern is for corn where the \(R^{2}\) for the Yang \& Zhang estimator
model increase from 18\% pre-financialization to 44\% during the
financialization phase.

Table 6 shows the results for the Granger causality analysis where we
observe a change in behavior between the two periods. For SRW wheat and
corn open interest Granger causes volatility (Yang \& Zhang estimator),
but not the other way round, in the pre-financialization period during
financialization period the only causal relation established is for SRW
wheat volatility (Yang \& Zhang estimator) Granger causing volume. In
contrast, over the same period, for HRW wheat, the smallest market, and
soybeans, volume appears to Granger cause volatility. There is thus no
clear pattern over the financialization period and the evidence is
stronger in the first period for volume Granger causing volatility.
Taken together these results provide further confirmation for a change
in relationship between volatility and both volume and open interest,
with some evidence that financialization may have driven volatility
higher in the smallest grain market.

The mean Working's T index levels range from 1.05 (HRW wheat) to 1.18
(SRW wheat) over the first period while they range from 1.08 (HRW wheat)
to 1.18 (SRW wheat) over the second period. The results indicate a very
limited mean level increase from the first to the second period, 1.9\%
on average across commodities, ranging from no change for SRW wheat to a
2.9\% increase for HRW wheat, suggesting that the level of ``excess
speculation'' as defined by Working (1960) did not uncouple with
historical levels during financialization\footnote{As suggested in
  Sanders and Irwin (2010).}.

Table 7 shows the results for the regressions of volatility on
contemporaneous and one-week lagged total open interest, open interest
of speculators, open interest of hedgers, volume and Working's T index.
The pattern of results is very similar for both the open interest of
hedgers and speculators in both periods indicating that the influence of
hedgers and speculators on volatility is very similar and did not change
over financialization. For SRW wheat, the slope coefficients for both
hedgers and speculators are positive in both periods indicating a
destabilizing effect, with a higher \(R^{2}\) in the second period
suggesting a greater destabilizing effect. The results for corn indicate
a change from stabilizing, with both coefficients negative
pre-financialization, to destabilizing with positive coefficients during
financialization accompanied by a sharp increase in \(R^{2}\). For HRW
wheat, the evidence is mixed with some evidence of a greater stabilizing
influence for both hedgers and speculators while for soybeans all slope
coefficients are negative but with lower \(R^{2}\) during
financialization indicating a weaker stabilizing influence. The overall
thrust of the evidence is towards a greater destabilizing influence of
open interest on volatility with little difference between hedgers and
speculators. These findings echo some of the issues raised in Cheng and
Xiong (2014b) and raises issues about the classification system used by
the CFTC. The results for basis volatility are similar except for HRW
wheat for which open interest shows a trend towards greater
destabilization influence on volatility. The results for the Working's T
index provide no clear support for the hypothesis that excess
speculation may have caused changes in the nature of commodity futures
volatility. The slope coefficient for corn switches from positive to
negative indicating that ``excess speculation'' seemed to have a
stabilizing influence on volatility during financialization. The results
are mixed for wheat with a switch from stabilizing to destabilizing for
SRW wheat and negative coefficients in both periods for HRW wheat albeit
with lower \(R^{2}\) over financialization while for soybeans there is
evidence of a greater destabilizing effect. These results further
reinforce the hypothesis that dispersion in beliefs or
disagreement-based trading (Stout, 2011) could be one of the driving
forces behind some of the effects observed over the financialization
phase. For basis volatility there is no clear pattern with five out of
eight coefficients negative for the financialization period.

Table 8 shows the results for a four-factor model where the response is
volatility and the explanatory variables include contemporaneous and
one-week lagged open interest of hedgers, open interest of speculators,
volume and Working's T index. The results are strongest for corn with a
dramatic increase in \(R^{2}\) (20.9\% to 52.2\%) while all the others
results are mixed, reinforcing the earlier findings above indicating
that the effects of financialization seem to have been most pronounced
in the corn market.

The results of the Granger causality analysis are shown in Table 9 and
are generally inconclusive with the only clear evidence for SRW wheat
where futures volatility and volume Granger cause each other and futures
volatility Granger causes open interest of hedgers over the
financialization period.

\newpage

\hypertarget{conclusions}{%
\section{Conclusion}\label{conclusions}}

Institutional investments in long-only commodity index funds soared in
the early 2000s which led to a large increase in trading activity as
measured by open interest and volume. This surge and its attendant
impacts are commonly referred to as the ``financialization'' of
commodity markets. In this paper, we investigate the effect of
financialization on the volatility of grain futures markets. These
markets have been classically regarded as hedging markets and have been
at the forefront of the debate over whether financialization has
contributed to price and volatility spikes. We find that all grains
futures volatility increased in the period of financialization but that
this increase is not as large as commonly believed. Nonetheless, we find
that the increase in market depth has a generally destabilizing effect
on grains volatility but that this destabilizing effect does not seem to
be driven by the action of speculators. Our results generally support
the findings of Stein (1987), Goldstein et al. (2014), Singleton (2013),
and Sockin and Xiong (2015) which suggest that increased market depth
that arises as a result of difference of opinion between traders could
lower the information content of prices. Our results therefore add to
the growing body of evidence that suggest that greater market depth may
not always have a beneficial effect on markets.

\newpage

\hypertarget{refs}{}
\leavevmode\hypertarget{ref-basak_model_2016}{}%
Basak, S., Pavlova, A., 2016. A model of financialization of
commodities. The Journal of Finance 71, 1511--1556.
doi:\href{https://doi.org/10.1111/jofi.12408}{10.1111/jofi.12408}

\leavevmode\hypertarget{ref-bessembinder_price_1993}{}%
Bessembinder, H., Seguin, P.J., 1993. Price volatility, trading volume,
and market depth: Evidence from futures markets. Journal of Financial
and Quantitative Analysis 28, 21--39.
doi:\href{https://doi.org/10.2307/2331149}{10.2307/2331149}

\leavevmode\hypertarget{ref-bohl_does_2013}{}%
Bohl, M.T., Stephan, P.M., 2013. Does futures speculation destabilize
spot prices? New evidence for commodity markets. Journal of Agricultural
and Applied Economics 45, 595--616.
doi:\href{https://doi.org/10.1017/S1074070800005150}{10.1017/S1074070800005150}

\leavevmode\hypertarget{ref-bos_bitter_2012}{}%
Bos, J.W., Molen, M. van der, 2012. A bitter brew? Futures speculation
and commodity prices. Futures Speculation and Commodity Prices.
doi:\href{https://doi.org/10.2139/ssrn.2209706}{10.2139/ssrn.2209706}

\leavevmode\hypertarget{ref-brunetti_is_2009}{}%
Brunetti, C., Buyuksahin, B., 2009. Is speculation destabilizing?
Working paper.
doi:\href{https://doi.org/10.2139/ssrn.1393524}{10.2139/ssrn.1393524}

\leavevmode\hypertarget{ref-buyuksahin_speculators_2014}{}%
Büyükşahin, B., Robe, M.A., 2014. Speculators, commodities and
cross-market linkages. Journal of International Money and Finance,
Understanding international commodity price fluctuations 42, 38--70.
doi:\href{https://doi.org/10.1016/j.jimonfin.2013.08.004}{10.1016/j.jimonfin.2013.08.004}

\leavevmode\hypertarget{ref-caballero_financial_2008}{}%
Caballero, R.J., Farhi, E., Gourinchas, P.-O., 2008. Financial crash,
commodity prices and global imbalances. National Bureau of Economic
Research.

\leavevmode\hypertarget{ref-cheng_financialization_2014}{}%
Cheng, I.-H., Xiong, W., 2014a. Financialization of commodity markets.
Annual Review of Financial Economics 6, 419--441.
doi:\href{https://doi.org/10.1146/annurev-financial-110613-034432}{10.1146/annurev-financial-110613-034432}

\leavevmode\hypertarget{ref-cheng_why_2014}{}%
Cheng, I.-H., Xiong, W., 2014b. Why do hedgers trade so much? The
Journal of Legal Studies 43, S183--S207.
doi:\href{https://doi.org/10.1086/675720}{10.1086/675720}

\leavevmode\hypertarget{ref-deschutter_food_2010}{}%
De Schutter, O., 2010. Food commodities speculation and food price
crises: Regulation to reduce the risks of price volatility. United
Nations Special Rapporteur on the Right to Food Briefing Note 2, 1--14.

\leavevmode\hypertarget{ref-domanski_financial_2007}{}%
Domanski, D., Heath, A., 2007. Financial investors and commodity
markets. Working Paper.

\leavevmode\hypertarget{ref-du_financial_2017}{}%
Du, D., Zhao, X., 2017. Financial investor sentiment and the boom/bust
in oil prices during 2003--2008. Review of Quantitative Finance and
Accounting 48, 331--361.
doi:\href{https://doi.org/10.1007/s11156-016-0553-5}{10.1007/s11156-016-0553-5}

\leavevmode\hypertarget{ref-fung_information_2003}{}%
Fung, H.-G., Leung, W.K., Xu, X.E., 2003. Information flows between the
u.s. And china commodity futures trading. Review of Quantitative Finance
and Accounting 21, 267--285.
doi:\href{https://doi.org/10.1023/A:1027384330827}{10.1023/A:1027384330827}

\leavevmode\hypertarget{ref-gannon_simultaneous_2010}{}%
Gannon, G.L., 2010. Simultaneous volatility transmission and spillover
effects. Review of Pacific Basin Financial Markets and Policies 13,
127--156.
doi:\href{https://doi.org/10.1142/S0219091510001895}{10.1142/S0219091510001895}

\leavevmode\hypertarget{ref-gilbert_speculative_2010}{}%
Gilbert, C.L., 2010a. Speculative influences on commodity futures prices
2006-2008. UNCTAD, Geneva.

\leavevmode\hypertarget{ref-gilbert_how_2010}{}%
Gilbert, C.L., 2010b. How to understand high food prices. Journal of
Agricultural Economics 61, 398--425.
doi:\href{https://doi.org/10.1111/j.1477-9552.2010.00248.x}{10.1111/j.1477-9552.2010.00248.x}

\leavevmode\hypertarget{ref-goldstein_speculation_2014}{}%
Goldstein, I., Li, Y., Yang, L., 2014. Speculation and hedging in
segmented markets. The Review of Financial Studies 27, 881--922.
doi:\href{https://doi.org/10.1093/rfs/hht059}{10.1093/rfs/hht059}

\leavevmode\hypertarget{ref-goldstein_information_2015}{}%
Goldstein, I., Yang, L., 2015. Information diversity and
complementarities in trading and information acquisition. The Journal of
Finance 70, 1723--1765.
doi:\href{https://doi.org/10.1111/jofi.12226}{10.1111/jofi.12226}

\leavevmode\hypertarget{ref-hamilton_causes_2009}{}%
Hamilton, J.D., 2009. Causes and consequences of the oil shock of
2007-08. National Bureau of Economic Research.

\leavevmode\hypertarget{ref-hamilton_effects_2015}{}%
Hamilton, J.D., Wu, J.C., 2015. Effects of index-fund investing on
commodity futures prices. International Economic Review 56, 187--205.
doi:\href{https://doi.org/10.1111/iere.12099}{10.1111/iere.12099}

\leavevmode\hypertarget{ref-harris_role_2009}{}%
Harris, J.H., Buyuksahin, B., 2009. The role of speculators in the crude
oil futures market. Working paper, CFTC.
doi:\href{https://doi.org/10.2139/ssrn.1435042}{10.2139/ssrn.1435042}

\leavevmode\hypertarget{ref-henderson_new_2015}{}%
Henderson, B.J., Pearson, N.D., Wang, L., 2015. New evidence on the
financialization of commodity markets. The Review of Financial Studies
28, 1285--1311.
doi:\href{https://doi.org/10.1093/rfs/hhu091}{10.1093/rfs/hhu091}

\leavevmode\hypertarget{ref-herman_not_2011}{}%
Herman, M.-O., Kelly, R., Nash, R., 2011. Not a game, speculation v.
Food security: Regulating financial markets to grow a better future.
Oxfam policy and practice: agriculture, food and land 11, 127--138.

\leavevmode\hypertarget{ref-irwin_index_2011}{}%
Irwin, S.H., Sanders, D.R., 2011. Index funds, financialization, and
commodity futures markets. Applied Economic Perspectives and Policy 33,
1--31.
doi:\href{https://doi.org/10.1093/aepp/ppq032}{10.1093/aepp/ppq032}

\leavevmode\hypertarget{ref-irwin_financialization_2012}{}%
Irwin, S.H., Sanders, D.R., 2012a. Financialization and structural
change in commodity futures markets. Journal of Agricultural and Applied
Economics 44, 371--396.
doi:\href{https://doi.org/10.1017/S1074070800000481}{10.1017/S1074070800000481}

\leavevmode\hypertarget{ref-irwin_testing_2012}{}%
Irwin, S.H., Sanders, D.R., 2012b. Testing the masters' hypothesis in
commodity futures markets. Energy economics 34, 256--269.
doi:\href{https://doi.org/10.1016/j.eneco.2011.10.008}{10.1016/j.eneco.2011.10.008}

\leavevmode\hypertarget{ref-irwin_devil_2009}{}%
Irwin, S.H., Sanders, D.R., Merrin, R.P., 2009. Devil or angel? The role
of speculation in the recent commodity price boom (and bust). Journal of
Agricultural and Applied Economics 41, 377--391.
doi:\href{https://doi.org/10.1017/S1074070800002856}{10.1017/S1074070800002856}

\leavevmode\hypertarget{ref-jiang_volatility_2017}{}%
Jiang, Y., Ahmed, S., Liu, X., 2017. Volatility forecasting in the
chinese commodity futures market with intraday data. Review of
Quantitative Finance and Accounting 48, 1123--1173.
doi:\href{https://doi.org/10.1007/s11156-016-0570-4}{10.1007/s11156-016-0570-4}

\leavevmode\hypertarget{ref-karpoff_relation_1987}{}%
Karpoff, J.M., 1987. The relation between price changes and trading
volume: A survey. Journal of Financial and Quantitative Analysis 22,
109--126. doi:\href{https://doi.org/10.2307/2330874}{10.2307/2330874}

\leavevmode\hypertarget{ref-kilian_role_2014}{}%
Kilian, L., Murphy, D.P., 2014. The role of inventories and speculative
trading in the global market for crude oil. Journal of Applied
Econometrics 29, 454--478.
doi:\href{https://doi.org/10.1002/jae.2322}{10.1002/jae.2322}

\leavevmode\hypertarget{ref-korniotis_does_2009}{}%
Korniotis, G.M., others, 2009. Does speculation affect spot price
levels?: The case of metals with and without futures markets. Division
of Research \& Statistics; Monetary Affairs, Federal Reserve Board.

\leavevmode\hypertarget{ref-krugman_more_2008}{}%
Krugman, P., 2008. More on oil and speculation. New York Times 1, 1--31.

\leavevmode\hypertarget{ref-kyle_continuous_1985}{}%
Kyle, A.S., 1985. Continuous auctions and insider trading. Econometrica
53, 1315--1335.
doi:\href{https://doi.org/10.2307/1913210}{10.2307/1913210}

\leavevmode\hypertarget{ref-li_when_2016}{}%
Li, J., 2016. When noise trading fades, volatility rises. Review of
Quantitative Finance and Accounting 47, 475--512.
doi:\href{https://doi.org/10.1007/s11156-015-0508-2}{10.1007/s11156-015-0508-2}

\leavevmode\hypertarget{ref-masters_testimony_2008}{}%
Masters, M.W., 2008. Testimony before the committee on homeland security
and governmental affairs. U.S. Senate 20.

\leavevmode\hypertarget{ref-masters_accidental_2011}{}%
Masters, M.W., White, A., 2011. The accidental hunt brothers: How
institutional investors are driving up food and energy prices: Executive
summary. Excessive Speculation in Agriculture Commodities 9--10.

\leavevmode\hypertarget{ref-nishina_nonlinear_2012}{}%
Nishina, K., Maghrebi, N., Holmes, M.J., 2012. Nonlinear adjustments of
volatility expectations to forecast errors: Evidence from markov-regime
switches in implied volatility. Review of Pacific Basin Financial
Markets and Policies 15, 1250007.
doi:\href{https://doi.org/10.1142/S0219091512500075}{10.1142/S0219091512500075}

\leavevmode\hypertarget{ref-petzel_testimony_2009}{}%
Petzel, T.E., 2009. Testimony before the commodity futures trading
commission.

\leavevmode\hypertarget{ref-phillips_testing_1990}{}%
Phillips, P.C., Loretan, M., 1990. Testing covariance stationarity under
moment condition failure with an application to common stock returns.
Cowles Foundation for Research in Economics, Yale University.

\leavevmode\hypertarget{ref-phillips_explosive_2011}{}%
Phillips, P.C., Wu, Y., Yu, J., 2011. Explosive behavior in the 1990s
nasdaq: When did exuberance escalate asset values? International
economic review 52, 201--226.
doi:\href{https://doi.org/10.1111/j.1468-2354.2010.00625.x}{10.1111/j.1468-2354.2010.00625.x}

\leavevmode\hypertarget{ref-phillips_dating_2011}{}%
Phillips, P.C., Yu, J., 2011. Dating the timeline of financial bubbles
during the subprime crisis. Quantitative Economics 2, 455--491.
doi:\href{https://doi.org/10.3982/QE82}{10.3982/QE82}

\leavevmode\hypertarget{ref-pirrong_restricting_2008}{}%
Pirrong, C., 2008. Restricting speculation will not reduce oil prices.
The Wall Street Journal 11.

\leavevmode\hypertarget{ref-pirrong_no_2010}{}%
Pirrong, C., 2010. No theory? No evidence? No problem! Regulation 33,
38.

\leavevmode\hypertarget{ref-rouwenhorst_commodity_2012}{}%
Rouwenhorst, K.G., Tang, K., 2012. Commodity investing. Annual Review of
Financial Economics 4, 447--467.
doi:\href{https://doi.org/10.1146/annurev-financial-110311-101716}{10.1146/annurev-financial-110311-101716}

\leavevmode\hypertarget{ref-sanders_marginal_2012}{}%
Sanders, D.J., Baker, T.G., 2012. Marginal hedging in futures markets.
Working paper.

\leavevmode\hypertarget{ref-sanders_futures_2008}{}%
Sanders, D.R., Irwin, S.H., 2008. Futures imperfect. New York Times 20.

\leavevmode\hypertarget{ref-sanders_speculative_2010}{}%
Sanders, D.R., Irwin, S.H., 2010. A speculative bubble in commodity
futures prices? Cross-sectional evidence. Agricultural Economics 41,
25--32.
doi:\href{https://doi.org/10.1111/j.1574-0862.2009.00422.x}{10.1111/j.1574-0862.2009.00422.x}

\leavevmode\hypertarget{ref-sanders_impact_2011}{}%
Sanders, D.R., Irwin, S.H., 2011. The impact of index funds in commodity
futures markets: A systems approach. The Journal of Alternative
Investments 14, 40--49.

\leavevmode\hypertarget{ref-sanders_adequacy_2010}{}%
Sanders, D.R., Irwin, S.H., Merrin, R.P., 2010. The adequacy of
speculation in agricultural futures markets: Too much of a good thing?
Applied Economic Perspectives and Policy 32, 77--94.
doi:\href{https://doi.org/10.1093/aepp/ppp006}{10.1093/aepp/ppp006}

\leavevmode\hypertarget{ref-schumann_hunger_2011}{}%
Schumann, H., 2011. The hunger-makers: How deutsche bank, goldman sachs
and other financial institutions are speculating with food at the
expense of the poorest. Foodwatch, Berlin.

\leavevmode\hypertarget{ref-senate_excessive_2009}{}%
Senate, U., 2009. Excessive speculation in the wheat market. Majority
and Minority Staff Report. Permanent Subcommittee on Investigations 24,
107--108.

\leavevmode\hypertarget{ref-singleton_investor_2013}{}%
Singleton, K.J., 2013. Investor flows and the 2008 boom/bust in oil
prices. Management Science 60, 300--318.
doi:\href{https://doi.org/10.1287/mnsc.2013.1756}{10.1287/mnsc.2013.1756}

\leavevmode\hypertarget{ref-smith_forecasting_2003}{}%
Smith, K.L., Bracker, K., 2003. Forecasting changes in copper futures
volatility with g.a.r.c.h. Models using an iterated algorithm. Review of
Quantitative Finance and Accounting 20, 245--265.
doi:\href{https://doi.org/10.1023/A:1023672428643}{10.1023/A:1023672428643}

\leavevmode\hypertarget{ref-sockin_informational_2015}{}%
Sockin, M., Xiong, W., 2015. Informational frictions and commodity
markets. The Journal of Finance 70, 2063--2098.
doi:\href{https://doi.org/10.1111/jofi.12261}{10.1111/jofi.12261}

\leavevmode\hypertarget{ref-stein_informational_1987}{}%
Stein, J., 1987. Informational externalities and welfare-reducing
speculation. Journal of Political Economy 95, 1123--1145.
doi:\href{https://doi.org/10.1086/261508}{10.1086/261508}

\leavevmode\hypertarget{ref-stoll_program_1987}{}%
Stoll, H.R., Whaley, R.E., 1987. Program trading and expiration-day
effects. Financial Analysts Journal 43, 16--28.
doi:\href{https://doi.org/10.2469/faj.v43.n2.16}{10.2469/faj.v43.n2.16}

\leavevmode\hypertarget{ref-stoll_commodity_2010}{}%
Stoll, H.R., Whaley, R.E., 2010. Commodity index investing and commodity
futures prices. Journal of Applied Finance 20.

\leavevmode\hypertarget{ref-stout_why_1998}{}%
Stout, L.A., 1998. Why the law hates speculators: Regulation and private
ordering in the market for o.t.c. Derivatives. Duke Law Journal 48,
701--786.

\leavevmode\hypertarget{ref-stout_uncertainty_2011}{}%
Stout, L.A., 2011. Uncertainty, dangerous optimism, and speculation: An
inquiry into some limits of democratic governance. Cornell Law Review
97, 1177.

\leavevmode\hypertarget{ref-tang_index_2012}{}%
Tang, K., Xiong, W., 2012. Index investment and the financialization of
commodities. Financial Analysts Journal 68, 54--74.
doi:\href{https://doi.org/10.2469/faj.v68.n6.5}{10.2469/faj.v68.n6.5}

\leavevmode\hypertarget{ref-till_has_2009}{}%
Till, H., 2009. Has there been excessive speculation in the u.s. Oil
futures markets? What can we (carefully) conclude from new cftc data?
Working paper.
doi:\href{https://doi.org/10.2139/ssrn.2608027}{10.2139/ssrn.2608027}

\leavevmode\hypertarget{ref-toda_statistical_1995}{}%
Toda, H.Y., Yamamoto, T., 1995. Statistical inference in vector
autoregressions with possibly integrated processes. Journal of
Econometrics 66, 225--250.
doi:\href{https://doi.org/10.1016/0304-4076(94)01616-8}{10.1016/0304-4076(94)01616-8}

\leavevmode\hypertarget{ref-unctad_global_2009}{}%
UNCTAD, S.T.F. on S.I., Cooperation, E., 2009. The global economic
crisis: Systemic failures and multilateral remedies. UNCTAD, Geneva.

\leavevmode\hypertarget{ref-wang_behavior_2003}{}%
Wang, C., 2003. The behavior and performance of major types of futures
traders. Journal of Futures Markets 23, 1--31.
doi:\href{https://doi.org/10.1002/fut.10056}{10.1002/fut.10056}

\leavevmode\hypertarget{ref-working_hedging_1953}{}%
Working, H., 1953. Hedging reconsidered. Journal of Farm Economics 35,
544--561. doi:\href{https://doi.org/10.2307/1233368}{10.2307/1233368}

\leavevmode\hypertarget{ref-working_whose_1954}{}%
Working, H., 1954. Whose markets? Evidence on some aspects of futurest
trading. Journal of Marketing 19, 1--11.
doi:\href{https://doi.org/10.1177/002224295401900101}{10.1177/002224295401900101}

\leavevmode\hypertarget{ref-working_speculation_1960}{}%
Working, H., 1960. Speculation on hedging markets. Food Research
Institute Studies.

\leavevmode\hypertarget{ref-working_new_1962}{}%
Working, H., 1962. New concepts concerning futures markets and prices.
The American Economic Review 52, 431--459.


\end{document}


