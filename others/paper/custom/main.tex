\documentclass[12pt,]{article}
\usepackage{lmodern}
\usepackage{amssymb,amsmath}
\usepackage{ifxetex,ifluatex}
\usepackage{fixltx2e} % provides \textsubscript
\ifnum 0\ifxetex 1\fi\ifluatex 1\fi=0 % if pdftex
  \usepackage[T1]{fontenc}
  \usepackage[utf8]{inputenc}
\else % if luatex or xelatex
  \ifxetex
    \usepackage{mathspec}
  \else
    \usepackage{fontspec}
  \fi
  \defaultfontfeatures{Ligatures=TeX,Scale=MatchLowercase}
\fi
% use upquote if available, for straight quotes in verbatim environments
\IfFileExists{upquote.sty}{\usepackage{upquote}}{}
% use microtype if available
\IfFileExists{microtype.sty}{%
\usepackage{microtype}
\UseMicrotypeSet[protrusion]{basicmath} % disable protrusion for tt fonts
}{}
\usepackage[letterpaper, left = 1in, right = 1in, top = 1.5in, bottom = 1.5in]{geometry}
\usepackage{hyperref}
\hypersetup{unicode=true,
            pdftitle={Financialization and Commodity Price Volatility: The Case of Grains},
            pdfkeywords={futures markets, agricultural commodities, financialization, risk,
speculation},
            pdfborder={0 0 0},
            breaklinks=true}
\urlstyle{same}  % don't use monospace font for urls
\usepackage{natbib}
\bibliographystyle{jf}
\IfFileExists{parskip.sty}{%
\usepackage{parskip}
}{% else
\setlength{\parindent}{0pt}
\setlength{\parskip}{6pt plus 2pt minus 1pt}
}
\setlength{\emergencystretch}{3em}  % prevent overfull lines
\providecommand{\tightlist}{%
  \setlength{\itemsep}{0pt}\setlength{\parskip}{0pt}}
\setcounter{secnumdepth}{5}
% Redefines (sub)paragraphs to behave more like sections
\ifx\paragraph\undefined\else
\let\oldparagraph\paragraph
\renewcommand{\paragraph}[1]{\oldparagraph{#1}\mbox{}}
\fi
\ifx\subparagraph\undefined\else
\let\oldsubparagraph\subparagraph
\renewcommand{\subparagraph}[1]{\oldsubparagraph{#1}\mbox{}}
\fi

%%% Use protect on footnotes to avoid problems with footnotes in titles
\let\rmarkdownfootnote\footnote%
\def\footnote{\protect\rmarkdownfootnote}


  \title{Financialization and Commodity Price Volatility: The Case of Grains}
    % \author{Devraj Basu \\ Olivier Bauthéac \\ Ameeta Jaiswal-Dale}
  \author{ Devraj Basu  \\  Olivier Bauthéac  \\  Ameeta Jaiswal-Dale  
	  \thanks{ Basu: University of Strathclyde, Glasgow, UK, \href{mailto:devraj.basu@strath.ac.uk}{\nolinkurl{devraj.basu@strath.ac.uk}}.  Bauthéac: University of Strathclyde, Glasgow, UK, \href{mailto:olivier.bautheac@strath.ac.uk}{\nolinkurl{olivier.bautheac@strath.ac.uk}}.  Jaiswal-Dale: University of St.~Thomas, Minneapolis, US, \href{mailto:a9jaiswal@stthomas.edu}{\nolinkurl{a9jaiswal@stthomas.edu}}. 	  
	   }
  }
    \date{}
  
\usepackage{longtable, tabu, pdflscape, booktabs, caption, setspace}
\doublespacing
\renewcommand*{\arraystretch}{0.65}


\begin{document}
\maketitle
\begin{abstract}
The early 2000s have witnessed a dramatic increase in long-only
institutional investments in commodity markets. This surge and its
accompanying effects are commonly referred to as the financialization of
commodity markets. This paper studies the impact of this phenomenon on
price volatility in the grain markets where we focus on CBOT corn, wheat
\& soybeans and KCBOT wheat. Our results suggest that increases in
trading volume and open interest, a consequence of financialization,
appear to have changed the nature of grains volatility and seem
consistent with the model of \citet{stein_informational_1987} and
\citet{goldstein_information_2015} where the entry of new traders could
lower the information content of price for existing traders. Our
findings further suggest that the increase in market depth has a
generally destabilizing effect on grains volatility which provides some
support for the concerns of regulators. However this destabilizing
effect does not seem to be driven by the action of speculators. Our
analysis is thus overall more supportive of
\citet{singleton_investor_2013} and \citet{stein_informational_1987} in
that disagreement and difference of opinion are more likely to have
caused changes in the nature of grains volatility than excess
speculation. 
\newline
JEL: QO2, G11, G23, G13
\newline
Keywords: futures markets, agricultural commodities, financialization, risk,
speculation

\end{abstract}


\newpage

\hypertarget{introduction}{%
\section{Introduction}\label{introduction}}

In the early 2000s, against a backdrop of a low yield environment and
poor stock market performance, the investment industry developed
financial products designed for providing individuals and institutions
with buy-side exposure to commodities through over-the-counter (OTC)
swaps, exchange-traded funds (ETFs), and exchange-traded notes (ETNs),
all of which are linked to popular commodity indexes such as the Goldman
Sachs Commodity Index (S\&P-GSCI). As these products grew in popularity
investment in long-only commodity index funds rapidly soared\footnote{Assets
  allocated to commodity index replication strategies grew from \$13bn
  in 2003 to over \$300bn in 2008 \citep{masters_accidental_2011}.}.
Some refer to this large inflow of mostly institutional capital and its
impact as the financialization of commodity markets
\citep{domanski_financial_2007}. The issue has had a wide impact on
areas ranging from financial\footnote{For example
  \citet{tang_index_2012}, \citet{singleton_investor_2013},
  \citet{basak_model_2016}, \citet{henderson_new_2015}.} and
agricultural\footnote{\citet{irwin_financialization_2012} study the
  impact on agricultural markets, \citet{irwin_index_2011} and
  \citet{hamilton_effects_2015} on commodity markets in general,
  \citet{buyuksahin_speculators_2014} study the oil market while
  \citet{korniotis_does_2009} considers the metals market.} economics to
public policy\footnote{The first responses to the 2007/2008 crisis of
  escalating food and energy prices took the form of policy reports,
  many of which reasoned that the growth of commodity index funds came
  along with an influx of largely speculative capital that was
  responsible for driving commodity prices beyond their historic highs
  \citep[\citet{unctad_global_2009}, \citet{herman_not_2011},
  \citet{deschutter_food_2010},
  \citet{schumann_hunger_2011}]{senate_excessive_2009}.} and revived the
long-standing ``adequacy of speculation'' debate. Agricultural
commodities have been the forefront of the controversy over whether
``excess speculation'', allegedly brought about by financialization, has
contributed to price spikes in commodity markets\footnote{\citet{irwin_index_2011},
  \citet{rouwenhorst_commodity_2012},
  \citet{cheng_financialization_2014}\} and \citet{bos_bitter_2012}
  survey the literature and summarize the policy and academic debates.}.
On one side, championed by hedge fund manager Michael Masters, are those
who argue that index investor driven buying pressure created a massive
bubble in commodity prices\footnote{This contention is commonly referred
  to as the Masters' Hypothesis.}, particularly in the grain and energy
markets \citep[\citet{masters_accidental_2011},
\citet{caballero_financial_2008}, \citet{petzel_testimony_2009},
\citet{hamilton_causes_2009},
\citet{du_financial_2017}]{masters_testimony_2008}. Others, advocated by
the academic duo formed by Dwight Sanders and Scott Irwin, are
dismissive of this contention and point out inconsistencies as well as
contradictory facts in the bubble arguments
\citep[\citet{pirrong_restricting_2008}, \citet{pirrong_no_2010},
\citet{sanders_futures_2008}, \citet{irwin_devil_2009},
\citet{korniotis_does_2009}, \citet{harris_role_2009},
\citet{till_has_2009}, \citet{stoll_commodity_2010}]{krugman_more_2008}.
A number of academic studies attempted to sort out which side of the
debate is correct using a variety of economic tools\footnote{While
  \citet{gilbert_speculative_2010}, \citet{gilbert_how_2010},
  \citet{phillips_explosive_2011}, \citet{phillips_dating_2011},
  \citet{tang_index_2012} are supportive of the bubble argument,
  \citet{harris_role_2009}, \citet{brunetti_is_2009},
  \citet{sanders_adequacy_2010}, \citet{stoll_commodity_2010},
  \citet{sanders_speculative_2010}, \citet{sanders_adequacy_2010},
  \citet{sanders_impact_2011}, \citet{irwin_testing_2012},
  \citet{buyuksahin_speculators_2014}, \citet{kilian_role_2014} dismiss
  it.}. The majority of these studies does not support, and some of them
even refute, the bubble hypothesis suggesting that there is no direct
link between commodity institutional investments and commodity prices.
Nonetheless, the impact of financialization on commodity price
volatility, in the grain markets in particular, is still a source of
concern both from an academic as well as a regulatory
perspective\footnote{Concerns over the consequences of financialization
  were behind Rule 76 FR 4752 issued by the U.S. Commodity Futures
  Trading Commission (CFTC) on January 26, 2011. This provision emanates
  from the Dodd-Frank Wall Street and Consumer Protection Act of 2010
  (Title VII, Section 737) that mandates the CFTC to use position limits
  to restrict the flow of speculative capital into a number of commodity
  markets. The Rule was approved in a close 3-2 vote and the ensuing
  rule-making process was extremely contentious with several
  commissioners expressing reservations about the lack of supporting
  evidence and the Rule also triggering thousands of comment letters as
  well as a lawsuit against the CFTC. See remarks of ex CFTC Chairman
  Gary Gensler before the International Monetary Fund Conference
  (\href{http://www.cftc.gov/PressRoom/SpeechesTestimony/opagensler-137}{www.cftc.gov})
  as well as remarks of Commissioner Bart Chilton
  (\href{http://www.cftc.gov/PressRoom/SpeechesTestimony/chiltonstatement022412}{www.cftc.gov}).}
and has not been thoroughly investigated yet\footnote{\citet{bohl_does_2013}
  is one of the few studies that investigate this issue.}.

In this study we examine the nature of this impact\footnote{Our paper
  fits into the broad area of modeling changes in volatility and
  volatility transmission. See for example:
  \citet{gannon_simultaneous_2010}, \citet{jiang_volatility_2017},
  \citet{li_when_2016}, \citet{fung_information_2003},
  \citet{smith_forecasting_2003}, \citet{nishina_nonlinear_2012}.}. We
focus on Chicago Board of Trade (CBOT) corn, soybeans and soft red
winter wheat (SRW) as well as Kansas City Board of Trade (KCBOT) hard
red winter wheat (HRW), four major global commodities of which the U.S.
are a major producer. As financialization is largely a U.S. based
phenomenon it seems appropriate to study its effect on them. Besides,
these markets are regarded as classic hedging markets where speculation
tends to follow hedging volume \citep[\citet{working_whose_1954},
\citet{working_speculation_1960}, \citet{working_new_1962},
\citet{sanders_marginal_2012}]{working_hedging_1953} and are thus good
candidates for assessing the impact of financialization, where
speculation plays a central role. Also, the prices of these commodities
tend to be driven by similar fundamentals as they can be substitutes
and/or complements on the production and use sides. This makes it easier
to isolate the effect of financialization by examining the differences
in volatility patterns before and during financialization. Finally,
corn, soybeans, HRW and SRW wheat are constituents of major commercial
commodity indexes and are thus particularly suitable in this context.

We study the volatility of the futures front month returns as well as
that of the basis for each individual commodity considered over the
1992-2007 period using a set of volatility estimators that includes the
classic ``close-to-close'' as well as a number of range-based estimators
that account for intra-day price action. We define the 1992-2003 period
as the pre-financialization phase and the 2003-2007 as the
financialization phase with the 2003 cut-off based on earlier
studies\footnote{Most earlier studies locate the onset of
  financialization around the 2003-2004 period
  \citep[\citet{cheng_financialization_2014},
  \citet{hamilton_effects_2015}, \citet{irwin_index_2011},
  \citet{irwin_financialization_2012}, \citet{irwin_testing_2012},
  \citet{tang_index_2012}]{basak_model_2016}.}. We find a moderate
increase in futures average volatility (from 10\% to 25\% depending on
the estimator) and a much larger increase in basis average volatility
(from 30\% to over 100\% depending on the estimator). Although uniform,
the increase in average futures volatility is perhaps not as high as
proponents of the Masters' Hypothesis might have believed while that for
basis volatility suggests potentially stronger effects due to
financialization.

The relationship between volatility, volume and open interest has been
investigated quite extensively in the finance literature with some
studies suggesting that increases trading volume and open interest, via
the entry of new traders, could have a stabilizing effect on volatility
\citep[\citet{kyle_continuous_1985},
\citet{stoll_program_1987}]{bessembinder_price_1993} while others,
particularly in some recent studies, argue it could have a destabilizing
effect \citep[\citet{singleton_investor_2013},
\citet{sockin_informational_2015}]{goldstein_information_2015}. We
investigate the issue by running regressions of volatility on
contemporaneous and lagged volume and open interest.

For futures volatility we observe an increase in \(R^{2}\) over the
financialization period in most cases, sometimes dramatically\footnote{The
  increase is largest for corn, followed by soybeans and HRW wheat.
  Results are mixed for SRW wheat.} as well as an overall increase in
slope coefficient\footnote{For SRW and HRW wheat both open interest and
  volume switch from negative (as predicted by
  \citet{kyle_continuous_1985}, \citet{stoll_program_1987}) to positive
  (as predicted by most informational models;
  \citet{karpoff_relation_1987}) during financialization. For corn, open
  interest follows the same pattern while volume shows a positive
  relationship other the first period which gets stronger during
  financialization. Soybeans show the opposite pattern with a negative
  correlation for both open interest and volume which becomes more
  negative during financialization.} for both open interest and volume,
contemporaneous and lagged. The pattern of results is similar for basis
volatility. These results suggest that the relationship between
volatility and both volume and open interest has changed and taken
together with the large increase in level for both variables, suggest
that financialization has affected grains volatility. Open interest and
volume seem to have switched from a stabilizing to a destabilizing role
during financialization for corn and both HRW and SRW wheat while for
soybeans, the opposite seems to have happened. Overall our results
suggest that increases in trading volume and open interest, a
consequence of financialization, appear to have had a destabilizing
influence on grain volatility and seem consistent with the models of
\citet{stein_informational_1987}\footnote{He argues that entry of new
  traders could lower the information content of price for existing
  traders through noise in their signals.}
\citet{goldstein_information_2015}\footnote{They argue that the negative
  information content effect is caused by the behaviors of those traders
  who are informed of the same information but respond to this
  information in opposite directions. Their model suggests that
  commodity financialization could make futures prices less informative
  (they refer to price informativeness as the amount of residual
  uncertainty uninformed traders face after conditioning on prices).},
\citet{singleton_investor_2013}\footnote{He notes that learning about
  economic fundamentals with heterogeneous information may induce
  excessive price volatility, drift in commodity prices, and a tendency
  towards booms and busts. He argues that under these conditions the
  flow of financial index investments into commodity markets may harm
  price discovery and social welfare.} and
\citet{sockin_informational_2015}\footnote{They develop a model to
  analyze information aggregation in commodity markets. Their analysis
  highlights important feedback effects of informational noise
  originating from supply shocks and futures market trading on commodity
  demand and spot prices.}.

We try to assess the impact of speculators by running similar
regressions with open interest of hedgers and speculators separately as
allowed by the U.S. Commodity Futures Trading Commission (CFTC)'s
Commitment of Traders (COT) reports data\footnote{A similar sort of
  analysis using the CFTC's classification was carried out in
  \citet{wang_behavior_2003}. See the CFTC's explanatory notes for
  details on the traders classification in the COT legacy format:
  \href{http://www.cftc.gov/MarketReports/CommitmentsofTraders/ExplanatoryNotes/index.htm}{www.cftc.gov}}.
The results for speculators suggest a move towards a greater
destabilizing effect for HRW wheat, corn and SRW wheat with no clear
results for soybeans. The pattern is very similar for hedgers open
interest which raises questions about the classification\footnote{The
  CFTC has now refined its classification and publishes a
  ``disaggregated'' COT report with data going back to June 2006. In
  this report, the commercial category is further subdivided into
  processors/merchants and swap dealers while the non-commercial
  category is split into money managers and other reportables (not
  captured in the other groups). See the CFTC's explanatory notes for
  details on the traders classification in the COT disaggregated format:
  \href{http://www.cftc.gov/MarketReports/CommitmentsofTraders/DisaggregatedExplanatoryNotes/index.htm}{www.cftc.gov}}
and perhaps the nature of hedgers' activities\footnote{See
  \citet{cheng_why_2014}.} and also suggests that these may have been
influenced by financialization. The pattern of results for basis
volatility is very similar.

We investigate the speculation issue further by running similar
regressions with the Working's T index, a measure of excess
speculation\footnote{This measure originates in
  \citet{working_speculation_1960} and is widely used in the
  agricultural economics literature to assess the impact of excess
  speculation.}. The results show no clear pattern across commodities or
periods with both positive and negative slope coefficients in the second
period. There is thus no clear indication that the change in
relationship between volatility and open interest has been driven by
excess speculation or an increase in speculative open interest. We
conclude with a Granger causality analysis that we implement in turn on
each pair of the above-mentioned variables. The results are generally
inconclusive with some evidence that futures volatility and volume
Granger cause each other and futures volatility Granger causes hedger's
open interest in the second period.

Overall our results point to an increase in grains volatility as well as
a clear change in the relationship between grains volatility and open
interest and volume as a result of financialization. However there is no
clear evidence that this change was driven by the actions of
speculators. Our findings provide some support for the concerns of
regulators but are overall more supportive of
\citet{goldstein_information_2015}, \citet{singleton_investor_2013} and
\citet{stein_informational_1987} in that disagreement and difference of
opinion\footnote{This issue is also of relevance in the recent legal
  literature \citep[\citet{stout_uncertainty_2011}]{stout_why_1998}
  where it is referred to as disagreement based trading based on
  differing subjective beliefs about future prices.
  \citet{stout_uncertainty_2011} shows how this may be viewed as a sort
  of market failure.} are more likely to have caused changes in the
nature of grains volatility than excess speculation.

The rest of the paper is organized as follow: the data and methods are
described in the next section with the results discussed in section
\ref{results}, while section \ref{conclusions} concludes.

\newpage

\hypertarget{data-methods}{%
\section{Data \& methods}\label{data-methods}}

We study Chicago Board of Trade (CBOT) corn, soybeans and soft red
winter wheat (SRW) as well as Kansas City Board of Trade (KCBOT) hard
red winter wheat (HRW). The futures contract market quotes as well as
volume and open interest data are from Bloomberg wile spot market price
time series are from the Minneapolis Grain Exchange (MGEX).

Futures prices are observed every trading day at close while the daily
values for cash prices are constructed as the average of high and low
prices for the day due to the lack of open and close quotes in the cash
markets. We consider the front month futures contract until the first
week of the maturity month at which date the position is switched to the
next most liquid contract. For each commodity we define open interest
(volume) as the sum of all traders' positions (trading volume) for all
contracts on the term structure up to a year ahead. We define the basis
as the difference between the futures and cash (spot) price as follows:

\[B_{i, t}=F_{i, t}-S_{i, t}\] \(B_{i, t}\equiv\) basis for commodity
\textit{i}, at time \textit{t}.\\
\(F_{i, t}\equiv\) futures price for commodity \textit{i}, at time
\textit{t}.\\
\(S_{i, t}\equiv\) spot price for commodity \textit{i}, at time
\textit{t}.

We observe the 1992-2007 period and define the 1992-2003 period as the
pre-financialization phase and the 2003-2007 as the financialization
phase (with the 2003 cut-off based on earlier studies\footnote{Most
  earlier studies locate the onset of financialization around the
  2003-2004 period \citep[\citet{cheng_financialization_2014},
  \citet{hamilton_effects_2015}, \citet{irwin_index_2011},
  \citet{irwin_financialization_2012}, \citet{irwin_testing_2012},
  \citet{tang_index_2012}]{basak_model_2016}.}), and study the two
periods independently.

For futures volatility we consider a set of estimators that includes the
classic ``close-to-close'' as well as five range-based stochastic
volatility estimators: Parkinson, Garman \& Klass, Rogers \& Satchell,
Garman \& Klass-Yang \& Zhang and Yang \& Zhang. The Parkinson estimator
estimates the volatility of the underlying based on high and low prices.
The Garman \& Klass estimator assumes Brownian motion with zero drift
and no opening jumps and is 7.4 times more efficient than the
``close-to-close'' estimator. The Rogers \& Satchell estimator allows
for non-zero drift, but assumes no opening jump while the Garman \&
Klass-Yang \& Zhang estimator, a modified version of the Garman \& Klass
estimator allows for opening jumps. The Yang \& Zhang has minimum
estimation error, and is independent of drift and opening gaps. It can
be interpreted as a weighted average of the Rogers \& Satchell
estimator, the ``close-open'' volatility, and the ``open-close''
volatility. For futures the results are presented for volatility when
estimated using the classic ``close-to-close'' estimator, as it is the
most commonly used, and the Yang \& Zhang estimator as it has the
minimum estimation error, with the rest of the results available from
the authors upon request. For basis volatility, the lack of open and
close quotes restricts the set to the classic ``close-to-close'' and the
Parkinson estimators. Yet, a straightforward construction of the
Parkinson estimator is unworkable because of the few occurrences of the
zero value in the time series of the basis leading to undefined
volatility observations. As a result, we define the Parkinson estimate
of basis volatility as the difference between that of the futures and
that of the spot.

We study the relationship between volatility, volume and open interest
with a set of factor models that include combinations of the latter. We
construct four one-factor models where volatility is the response and
both contemporaneous and one-week lagged open interest and volume are in
turn the explanatory variable. We also construct four two-factor models
where volatility is the response and the explanatory variables are in
turn, contemporaneous and one-week lagged open interest, contemporaneous
and one-week lagged volume, open interest and volume as well as one-week
lagged open interest and one-week lagged volume. We conclude with a
four-factor model that includes all the above-mentioned as explanatory
variables.

The weekly CFTC COT report's format breaks down total open interest into
three categories of traders, namely hedgers (commercial), speculators
(non-commercial) and non-reportable with the latter gathering the
remaining traders who do not fit in the previous two categories. This
break down allows the construction of the Working's T index, a measure
of excess speculation, as follows:

\[
T_{i, t} =
\left\{\begin{matrix}
1+\frac{SS_{i, t}}{HS_{i, t}+HL_{i, t}} & if & HS_{i, t} \geq HL_{i, t}\\ 
1+\frac{SL_{i, t}}{HS_{i, t}+HL_{i, t}} & if & HS_{i, t} < HL_{i, t}
\end{matrix}\right.
\]

\(T_{i, t}\equiv\) Working's T index for commodity \textit{i}, at time
\textit{t}.\\
\(SS_{i, t}\equiv\) number of speculative short positions for commodity
\textit{i}, at time \textit{t}.\\
\(SL_{i, t}\equiv\) number of speculative long positions for commodity
\textit{i}, at time \textit{t}.\\
\(HS_{i, t}\equiv\) number of short hedging positions for commodity
\textit{i}, at time \textit{t}.\\
\(HL_{i, t}\equiv\) number of long hedging positions for commodity
\textit{i}, at time \textit{t}.

We refine the analysis with a set of one factor models where volatility
is the response and total open interest, open interest of hedgers, open
interest of speculators, volume and Working's T index are in turn the
explanatory variable. We conclude with two four-factor models with the
first one including the contemporaneous time series of all the above and
the second one their one-week lagged version.

We look deeper into the speculation issue with a careful Granger
causality analysis. \citet{phillips_testing_1990} showed that Granger
causality testing can be unreliable in a context similar to
financialization. We hence rely on a modified version of the procedure
developed by \citet{toda_statistical_1995} that accounts for
non-stationary time series. We implement the analysis at the daily
frequency with futures volatility as measured by the Yang \& Zhang
estimator, open interest and volume as well as at the weekly frequency
where we also include open interest of hedgers and speculators
separately.

Robustness considerations lead us to implement the whole analysis above
with a 1999 cut-off\footnote{Some studies date the premise of
  financialization back to the very late 1990s/early 2000s
  \citep[\citet{stoll_commodity_2010},
  \citet{buyuksahin_speculators_2014}]{bohl_does_2013}.}. The pattern of
results is similar although not as pronounced as with the 2003 cut-off
suggesting that the effects of financialization were strongest over the
2003-2007 period as suggested in earlier studies.

\newpage

\hypertarget{results}{%
\section{Results \& discussion}\label{results}}

Table 1 shows the mean futures front month and spot price levels, open
interest and volume for the four commodities and two periods of
interest. We observe a moderate increase of grain prices over the
financialization period with an average mean futures price increase of
11.7\%; 10.5\% for mean spot price. SRW wheat shows the strongest
increase with +17.2\% for mean futures price and +20.3\% for mean spot
price while corn shows the lowest with a 2\% and 0.8\% increase
respectively. For the basis the situation is not as clear with the wheat
complex showing a relative stability in numerical terms (+\$1.3 and
+\$0.4 respectively) while corn and soybeans show dramatic increases
(+\$6.4 or +138.5\% and +\$8.7 or +264.7\% respectively). In contrast to
prices, mean open interest and volume show a dramatic increase across
the board with average increase of 95.4\% and 63.5\% respectively
suggesting that the large increase of trading activity during
financialization did not lead to a commensurate increase in grain
prices.

Table 2 shows the average volatility levels for the four commodities and
the two periods of interest. For futures average volatility, estimators
include the classic ``close-to-close'' as well as the values for the
five range-based estimators. The pattern of results is different across
the various commodities and estimators but in all cases the average
volatility is higher over the financialization period with the average
mean increase for the set of estimators ranging from 11.3\% for SRW
wheat to 20.1\% for HRW wheat. For basis average volatility, estimators
include the classic ``close-to-close'' as well as Parkinson and show a
dramatic increase over the financialization with the average increase
for the two estimators ranging from 41.7\% for SRW wheat to 155.6\% for
corn.

Table 3 shows the results for the regressions of volatility on
contemporaneous and one-week lagged volume and open interest. For
futures volatility, the slope coefficients of open interest for corn and
both types of wheat change to positive during financialization from
negative before indicating open interest changes from having a
stabilizing effect on volatility in the first period consistent with a
classic hedging market
\citep[\citet{working_new_1962}]{working_speculation_1960} to a
destabilizing effect in the second for these commodities. The \(R^{2}\)s
vary across the two periods with a sharp increase for corn (11\% to
44\%) and a decrease for HRW wheat (28\% to 5\%) suggesting that while
the nature of the effect is uniform for the three commodities, the
magnitude is not. For soybeans the sign of all slope coefficients is
negative in both periods and the \(R^{2}\) is higher in the
financialization period indicating a greater stabilizing effect. The
pattern and magnitude of the results is very similar when lagged open
interest is used, suggesting that it is the expected component of open
interest that is driving these results \citep{bessembinder_price_1993}.
For volume, the slope coefficients switch from negative to positive for
SRW \& HRW wheat while, in both periods, they are positive for corn with
a substantially higher \(R^{2}\) over the financialization period (2.8\%
to 24.2\% for the Yang \& Zhang estimator) and negative for soybeans
with again a higher \(R^{2}\) in the financialization phase. The pattern
is very similar for lagged volume indicating that the results are driven
by expected volume. The pattern of results is thus similar for both open
interest and volume and indicates that over financialization they had a
destabilizing effect on futures volatility for corn and the two types of
wheat while having a more stabilizing effect for soybeans. Besides,
soybeans showed the most pronounced increase in average futures
volatility indicating that changes in the nature of volatility and the
changes in the magnitude of volatility are not necessarily directly
linked. Overall these findings provide evidence for the hypotheses
advanced in \citet{goldstein_speculation_2014},
\citet{singleton_investor_2013} and \citet{stein_informational_1987}
where more trading arising from greater dispersions of beliefs or the
entry of new traders increasing noise in signals could lead to changes
in the nature of volatility and the information content of prices. Thus
greater market depth may not always lead to more informative price
signals and this may be due to the nature of trading (see also
\citet{stout_why_1998} \& \citet{stout_uncertainty_2011}) for a
discussion of this issue from a legal standpoint). For basis volatility
there does not appear to be a clear pattern and the results suggest that
the nature of basis volatility is different from that of futures, with
other factors perhaps driving the increases.

Combining volume and open interest produces a mixed pattern of results
as shown in table 4. For futures volatility, corn and SRW wheat show a
change in sign and increase in \(R^{2}\) (Yang \& Zhang estimator)
during financialization while for HRW wheat, there is a change in sign
for the open interest slope coefficient with the coefficient for volume
positive in both periods and a lower \(R^{2}\) for the financialization
period. These results suggest that both factors jointly had a
destabilizing influence on volatility. For soybeans, the two
coefficients are negative in both periods with a higher \(R^{2}\) for
the financialization period indicating a greater stabilizing effect. The
results for the four-factor model are shown in table 5. The clearest
pattern is for corn where the \(R^{2}\) for the Yang \& Zhang estimator
model increase from 18\% pre-financialization to 44\% during the
financialization phase.

Table 6 shows the results for the Granger causality analysis where we
observe a change in behavior between the two periods. For SRW wheat and
corn open interest Granger causes volatility (Yang \& Zhang estimator),
but not the other way round, in the pre-financialization period during
financialization period the only causal relation established is for SRW
wheat volatility (Yang \& Zhang estimator) Granger causing volume. In
contrast, over the same period, for HRW wheat, the smallest market, and
soybeans, volume appears to Granger cause volatility. There is thus no
clear pattern over the financialization period and the evidence is
stronger in the first period for volume Granger causing volatility.
Taken together these results provide further confirmation for a change
in relationship between volatility and both volume and open interest,
with some evidence that financialization may have driven volatility
higher in the smallest grain market.

The mean Working's T index levels range from 1.05 (HRW wheat) to 1.18
(SRW wheat) over the first period while they range from 1.08 (HRW wheat)
to 1.18 (SRW wheat) over the second period. The results indicate a very
limited mean level increase from the first to the second period, 1.9\%
on average across commodities, ranging from no change for SRW wheat to a
2.9\% increase for HRW wheat, suggesting that the level of ``excess
speculation'' as defined by Working (1960) did not uncouple with
historical levels during financialization\footnote{As suggested in
  \citet{sanders_speculative_2010}.}.

Table 7 shows the results for the regressions of volatility on
contemporaneous and one-week lagged total open interest, open interest
of speculators, open interest of hedgers, volume and Working's T index.
The pattern of results is very similar for both the open interest of
hedgers and speculators in both periods indicating that the influence of
hedgers and speculators on volatility is very similar and did not change
over financialization. For SRW wheat, the slope coefficients for both
hedgers and speculators are positive in both periods indicating a
destabilizing effect, with a higher \(R^{2}\) in the second period
suggesting a greater destabilizing effect. The results for corn indicate
a change from stabilizing, with both coefficients negative
pre-financialization, to destabilizing with positive coefficients during
financialization accompanied by a sharp increase in \(R^{2}\). For HRW
wheat, the evidence is mixed with some evidence of a greater stabilizing
influence for both hedgers and speculators while for soybeans all slope
coefficients are negative but with lower \(R^{2}\) during
financialization indicating a weaker stabilizing influence. The overall
thrust of the evidence is towards a greater destabilizing influence of
open interest on volatility with little difference between hedgers and
speculators. These findings echo some of the issues raised in
\citet{cheng_why_2014} and raises issues about the classification system
used by the CFTC. The results for basis volatility are similar except
for HRW wheat for which open interest shows a trend towards greater
destabilization influence on volatility. The results for the Working's T
index provide no clear support for the hypothesis that excess
speculation may have caused changes in the nature of commodity futures
volatility. The slope coefficient for corn switches from positive to
negative indicating that ``excess speculation'' seemed to have a
stabilizing influence on volatility during financialization. The results
are mixed for wheat with a switch from stabilizing to destabilizing for
SRW wheat and negative coefficients in both periods for HRW wheat albeit
with lower \(R^{2}\) over financialization while for soybeans there is
evidence of a greater destabilizing effect. These results further
reinforce the hypothesis that dispersion in beliefs or
disagreement-based trading \citep{stout_uncertainty_2011} could be one
of the driving forces behind some of the effects observed over the
financialization phase. For basis volatility there is no clear pattern
with five out of eight coefficients negative for the financialization
period.

Table 8 shows the results for a four-factor model where the response is
volatility and the explanatory variables include contemporaneous and
one-week lagged open interest of hedgers, open interest of speculators,
volume and Working's T index. The results are strongest for corn with a
dramatic increase in \(R^{2}\) (20.9\% to 52.2\%) while all the others
results are mixed, reinforcing the earlier findings above indicating
that the effects of financialization seem to have been most pronounced
in the corn market.

The results of the Granger causality analysis are shown in Table 9 and
are generally inconclusive with the only clear evidence for SRW wheat
where futures volatility and volume Granger cause each other and futures
volatility Granger causes open interest of hedgers over the
financialization period.

\newpage

\hypertarget{conclusions}{%
\section{Conclusion}\label{conclusions}}

Institutional investments in long-only commodity index funds soared in
the early 2000s which led to a large increase in trading activity as
measured by open interest and volume. This surge and its attendant
impacts are commonly referred to as the ``financialization'' of
commodity markets. In this paper, we investigate the effect of
financialization on the volatility of grain futures markets. These
markets have been classically regarded as hedging markets and have been
at the forefront of the debate over whether financialization has
contributed to price and volatility spikes. We find that all grains
futures volatility increased in the period of financialization but that
this increase is not as large as commonly believed. Nonetheless, we find
that the increase in market depth has a generally destabilizing effect
on grains volatility but that this destabilizing effect does not seem to
be driven by the action of speculators. Our results generally support
the findings of \citet{stein_informational_1987},
\citet{goldstein_speculation_2014}, \citet{singleton_investor_2013}, and
\citet{sockin_informational_2015} which suggest that increased market
depth that arises as a result of difference of opinion between traders
could lower the information content of prices. Our results therefore add
to the growing body of evidence that suggest that greater market depth
may not always have a beneficial effect on markets.

\newpage

\bibliography{C:/Users/yyb14214/Dropbox/research/literature/references.bib}


\end{document}
